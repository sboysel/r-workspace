\documentclass{article}\usepackage[]{graphicx}\usepackage[]{color}
%% maxwidth is the original width if it is less than linewidth
%% otherwise use linewidth (to make sure the graphics do not exceed the margin)
\makeatletter
\def\maxwidth{ %
  \ifdim\Gin@nat@width>\linewidth
    \linewidth
  \else
    \Gin@nat@width
  \fi
}
\makeatother

\definecolor{fgcolor}{rgb}{0.345, 0.345, 0.345}
\newcommand{\hlnum}[1]{\textcolor[rgb]{0.686,0.059,0.569}{#1}}%
\newcommand{\hlstr}[1]{\textcolor[rgb]{0.192,0.494,0.8}{#1}}%
\newcommand{\hlcom}[1]{\textcolor[rgb]{0.678,0.584,0.686}{\textit{#1}}}%
\newcommand{\hlopt}[1]{\textcolor[rgb]{0,0,0}{#1}}%
\newcommand{\hlstd}[1]{\textcolor[rgb]{0.345,0.345,0.345}{#1}}%
\newcommand{\hlkwa}[1]{\textcolor[rgb]{0.161,0.373,0.58}{\textbf{#1}}}%
\newcommand{\hlkwb}[1]{\textcolor[rgb]{0.69,0.353,0.396}{#1}}%
\newcommand{\hlkwc}[1]{\textcolor[rgb]{0.333,0.667,0.333}{#1}}%
\newcommand{\hlkwd}[1]{\textcolor[rgb]{0.737,0.353,0.396}{\textbf{#1}}}%

\usepackage{framed}
\makeatletter
\newenvironment{kframe}{%
 \def\at@end@of@kframe{}%
 \ifinner\ifhmode%
  \def\at@end@of@kframe{\end{minipage}}%
  \begin{minipage}{\columnwidth}%
 \fi\fi%
 \def\FrameCommand##1{\hskip\@totalleftmargin \hskip-\fboxsep
 \colorbox{shadecolor}{##1}\hskip-\fboxsep
     % There is no \\@totalrightmargin, so:
     \hskip-\linewidth \hskip-\@totalleftmargin \hskip\columnwidth}%
 \MakeFramed {\advance\hsize-\width
   \@totalleftmargin\z@ \linewidth\hsize
   \@setminipage}}%
 {\par\unskip\endMakeFramed%
 \at@end@of@kframe}
\makeatother

\definecolor{shadecolor}{rgb}{.97, .97, .97}
\definecolor{messagecolor}{rgb}{0, 0, 0}
\definecolor{warningcolor}{rgb}{1, 0, 1}
\definecolor{errorcolor}{rgb}{1, 0, 0}
\newenvironment{knitrout}{}{} % an empty environment to be redefined in TeX

\usepackage{alltt}
\usepackage{amsmath}
\usepackage{amssymb}
\usepackage{verbatim}
\usepackage{hyperref}
\usepackage{pdflscape}
\hypersetup{colorlinks=true,
            linkcolor=blue}
\usepackage[margin=1in]{geometry}
\IfFileExists{upquote.sty}{\usepackage{upquote}}{}
\begin{document}



\tableofcontents

\listoffigures

\section{Math} 
Here are some brilliant maths:
\[A\sum_{i = 0}^{\infty}y_{i}x_{i}^{2} \]
Please also see Figure \ref{fig:fig1}. Another cool plot can be seen
in Figure \ref{fig:fig2}.  See also, Figure \ref{fig:fig3}.  Consider even more
maths:
\[ \mathcal{B}_{\epsilon}(\mathbf{u}) = \{ \mathbf{x} \in \mathbb{R} | \lVert
\mathbf{x} - \mathbf{u} \rVert < \epsilon\}\]
I would also like to express the solution to the least squares problem:
\[\hat{\beta} = (\mathbf{X}'\mathbf{X})^{-1}\mathbf{X}'Y \]
Hopefully, this works.  Here is a change.  Hooray!  Now everytime I save my
\texttt{.Rnw} file, \texttt{knitr} and \texttt{latexmk} are called by
\texttt{watchman} from my \texttt{Makefile}.

\section{Code}
\begin{knitrout}
\definecolor{shadecolor}{rgb}{0.969, 0.969, 0.969}\color{fgcolor}\begin{kframe}
\begin{alltt}
\hlkwd{sample}\hlstd{(LETTERS[}\hlnum{1}\hlopt{:}\hlnum{5}\hlstd{],} \hlnum{10}\hlstd{,} \hlkwc{replace} \hlstd{=} \hlnum{TRUE}\hlstd{)}
\end{alltt}
\begin{verbatim}
##  [1] "B" "A" "B" "E" "A" "B" "A" "E" "C" "B"
\end{verbatim}
\end{kframe}
\end{knitrout}

\begin{knitrout}
\definecolor{shadecolor}{rgb}{0.969, 0.969, 0.969}\color{fgcolor}\begin{kframe}
\begin{alltt}
\hlstd{hello_world} \hlkwb{<-} \hlkwa{function}\hlstd{() \{}
  \hlkwd{print}\hlstd{(}\hlstr{'Hello World!'}\hlstd{)}
\hlstd{\}}
\hlkwd{hello_world}\hlstd{()}
\end{alltt}
\begin{verbatim}
## [1] "Hello World!"
\end{verbatim}
\end{kframe}
\end{knitrout}

\begin{knitrout}
\definecolor{shadecolor}{rgb}{0.969, 0.969, 0.969}\color{fgcolor}\begin{kframe}
\begin{alltt}
\hlkwd{head}\hlstd{(rw.df)}
\end{alltt}
\begin{verbatim}
##          path period phase
## 1  0.27538280      1     A
## 2 -1.42617511      2     A
## 3 -0.91183677      3     A
## 4  0.32759316      4     A
## 5  0.97533127      5     A
## 6 -0.08553271      6     A
\end{verbatim}
\end{kframe}
\end{knitrout}

\clearpage
\section{Figures}

\begin{knitrout}
\definecolor{shadecolor}{rgb}{0.969, 0.969, 0.969}\color{fgcolor}\begin{kframe}
\begin{alltt}
\hlkwd{ggplot}\hlstd{(}\hlkwc{data} \hlstd{= data,} \hlkwd{aes}\hlstd{(}\hlkwc{x} \hlstd{= V1,} \hlkwc{fill} \hlstd{= V4))} \hlopt{+}
  \hlkwd{geom_histogram}\hlstd{(}\hlkwc{colour} \hlstd{=} \hlstr{'white'}\hlstd{)} \hlopt{+}
  \hlkwd{facet_wrap}\hlstd{(}\hlopt{~} \hlstd{V4)}
\end{alltt}
\end{kframe}\begin{figure}[ht]

{\centering \includegraphics[width=\maxwidth]{figure/fig1-1} 

}

\caption[Some Distributions]{Some Distributions}\label{fig:fig1}
\end{figure}


\end{knitrout}

\begin{knitrout}
\definecolor{shadecolor}{rgb}{0.969, 0.969, 0.969}\color{fgcolor}\begin{kframe}
\begin{alltt}
\hlkwd{ggplot}\hlstd{(}\hlkwc{data} \hlstd{= data,} \hlkwd{aes}\hlstd{(}\hlkwc{x} \hlstd{= V1,} \hlkwc{y} \hlstd{= V2,} \hlkwc{size} \hlstd{= V5,} \hlkwc{fill} \hlstd{= V4))} \hlopt{+}
  \hlkwd{geom_point}\hlstd{(}\hlkwc{colour} \hlstd{=} \hlstr{'white'}\hlstd{,} \hlkwc{alpha} \hlstd{=} \hlnum{0.8}\hlstd{,} \hlkwc{pch} \hlstd{=} \hlnum{21}\hlstd{)}
\end{alltt}
\end{kframe}\begin{figure}[ht]

{\centering \includegraphics[width=\maxwidth]{figure/fig2-1} 

}

\caption[Some Scatterplots]{Some Scatterplots}\label{fig:fig2}
\end{figure}


\end{knitrout}

\clearpage
\begin{knitrout}
\definecolor{shadecolor}{rgb}{0.969, 0.969, 0.969}\color{fgcolor}\begin{kframe}
\begin{alltt}
\hlkwd{map}\hlstd{(}\hlstr{'state'}\hlstd{,} \hlkwc{projection} \hlstd{=} \hlstr{'albers'}\hlstd{,} \hlkwc{par} \hlstd{=} \hlkwd{c}\hlstd{(}\hlkwc{lat0} \hlstd{=} \hlnum{30}\hlstd{,} \hlkwc{lat1} \hlstd{=} \hlnum{40}\hlstd{))}
\end{alltt}
\end{kframe}\begin{figure}[ht]

{\centering \includegraphics[width=\maxwidth]{figure/fig3-1} 

}

\caption[A Map]{A Map}\label{fig:fig3}
\end{figure}


\end{knitrout}

\clearpage
\begin{knitrout}
\definecolor{shadecolor}{rgb}{0.969, 0.969, 0.969}\color{fgcolor}\begin{kframe}
\begin{alltt}
\hlkwd{ggplot}\hlstd{(}\hlkwc{data} \hlstd{= rw.df)} \hlopt{+}
  \hlkwd{geom_line}\hlstd{(}\hlkwd{aes}\hlstd{(}\hlkwc{x} \hlstd{= period,} \hlkwc{y} \hlstd{= path,} \hlkwc{colour} \hlstd{= phase))}
\end{alltt}
\end{kframe}\begin{figure}[ht]

{\centering \includegraphics[width=\maxwidth]{figure/fig4-1} 

}

\caption[A Random Walk]{A Random Walk}\label{fig:fig4}
\end{figure}


\end{knitrout}

\end{document}
